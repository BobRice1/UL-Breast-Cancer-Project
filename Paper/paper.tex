\documentclass[reprint, aps, prl]{revtex4-2}

\usepackage{graphicx}
\usepackage{amsmath}
\usepackage{amssymb}
\graphicspath{{../Figures/}}

\begin{document}

\title{Unsupervised Learning Reveals Structure in Breast Cancer Tumour Data}

\author{Gillian Macdonald}
\author{Lenka Okasova}
\author{Bob Rice}
\affiliation{School of Physics and Astronomy, University of Nottingham, UK}

\begin{abstract}
Unsupervised learning methods offer a systematic approach to uncover latent structure in high-dimensional biomedical data without reliance on diagnostic labels. We analyse the Breast Cancer Wisconsin dataset using principal component analysis (PCA) followed by k-means clustering to probe whether malignant and benign tumour classes emerge naturally from feature statistics alone. Correlation analysis reveals a strongly coupled cluster of morphological and nuclear features associated with malignancy, motivating dimensionality reduction. PCA shows that the first two components capture 74.2\% of the total variance (80.2\% with three components), with clear separation between tumour classes in the reduced space. Applying k-means clustering in this low-dimensional representation yields robust unsupervised class recovery, achieving and adjusted Rand index of 0.84 relative to known tumour classes. These results demonstrate that correlated cytological abnormalities organise the data into a low-dimensional structure that cleanly separates malignant and benign tumours, highlighting the utility of simple unsupervised approaches for structure recovery in data. 
\end{abstract}
\maketitle

\end{document}